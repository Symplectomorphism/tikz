\date{}
\usepackage{amsmath,amsthm,amssymb}
\usepackage{xcolor}
\usepackage{tikz}
\usepackage[siunitx]{circuitikz}
\usepackage{tikz-3dplot}
\usepackage{ifthen}
\tikzset{isometricXYZ/.style={x={(-0.866cm,-0.5cm)}, y={(0.866cm,-0.5cm)}, z={(0cm,1cm)}}}
\usetikzlibrary{arrows,decorations.pathmorphing,positioning,fit,trees,shapes,automata,calc,intersections,decorations.markings,patterns,graphs,quotes,plotmarks}
%\usepackage{algorithm}
%\usepackage{algorithmic}
\newcommand{\R}{\mathbb{R}}
\newcommand{\F}{\mathbb{F}}
\newcommand{\bmat}[1]{\begin{bmatrix}#1\end{bmatrix}}
\newcommand{\xth}[1]{{#1}^{\mathrm{th}}}
\newcommand{\pd}[2]{\frac{\partial #1}{\partial #2} }
\newcommand{\pdd}[2]{\frac{\partial^2 #1}{\partial #2^2}}
\newcommand{\myitemsep}{\setlength\itemsep{-0.25em}}
\newcommand{\bigpar}[1]{\left( #1\right)}
\tikzset{myedge/.style={ thick,->,>=stealth',inner sep=0pt,outer sep=3pt}}
\tikzset{hv path/.style= {to path={-| (\tikztotarget)}}}
\tikzset{vh path/.style= {to path={|- (\tikztotarget)}}}

%%%%%%%%%% Using shift and rotate around:

\newcommand{\spring}[4]{
\path[shift={#1},rotate={#2},fill = white,opacity=1.0] (-0.25,-0.5) rectangle (0.25,0.5); % cover
\draw[thick,shift={#1},rotate={#2}] (0,0.5) -- +(0.25,-0.1) -- +(-0.25,-0.2)-- +(0.25,-0.3)-- +(-0.25,-0.4)-- +(0.25,-0.5)-- +(-0.25,-0.6)-- +(0.25,-0.7)-- +(-0.25,-0.8)-- +(0.25,-0.9) -- +(0,-1.0);
\draw[thick,shift={#1},rotate={#2}] (#3,0) node {#4};
}
\newcommand{\ground}[3]{
\draw[thick,shift={#1},rotate={#2}] (-0.5*#3cm,0) -- (0.5*#3cm,0);
\path[thick,shift={#1},rotate={#2},pattern=north west lines] (-0.5*#3cm,0) rectangle (0.5*#3cm,0.2);
}
\newcommand{\dashpot}[4]{
\path[shift={#1},rotate={#1},fill = white,opacity=1.0] (-0.25,0) rectangle (0.25,0.15);
\path[shift={#1},rotate={#1},draw,thick] (0,0) -- +(0.25,0) -- +(0.25,0.25) +(0.15,0.15) -- +(-0.15,0.15)  +(-0.25,0.25) -- +(-0.25,0) -- +(0,0);
\draw[thick,shift={#1},rotate={#1}] (#3,0) node {#4};
}
\newcommand{\mydashpot}[5]{
\begin{scope}[xshift=#1cm,yshift=#2cm,rotate=#3]
	\path[fill = white,opacity=1.0] (-0.25,0) rectangle (0.25,0.15);
	\path[draw,thick] (0,0) -- +(0.25,0) -- +(0.25,0.25) +(0.15,0.15) -- +(-0.15,0.15)  +(-0.25,0.25) -- +(-0.25,0) -- +(0,0);
	\node at (#4,0) {#5};
	\end{scope}
}
\newcommand{\lapof}[1]{\mathcal L \left\{ #1 \right\}}
\newcommand{\lapinv}[1]{\mathcal L^{-1} \left\{ #1 \right\}}
\newcommand{\evalat}[2]{\left. #1 \right|_{#2}}
\newcommand{\myarr}[3]{(-#2:#1) arc (-#2:#2:#1) node[#3] }
\newcommand{\mc}[1]{\mathcal{#1}}
\newcommand{\mred}[1]{\textcolor{red}{[#1]}}
\newcommand{\myco}[2]{\bigpar{ \frac{#1}{#2} }}
\newcommand{\tc}[2]{{#1}{#2}}
\newcommand{\mcb}[1]{{\color{blue}#1}}
\newcommand{\mcr}[1]{{\color{red}#1}}
\newcommand{\mcg}[1]{{\color{green!70!black}#1}}
\usepackage{hyperref}
\hypersetup{colorlinks=true,
linkcolor=blue,          % color of internal links
        citecolor=green,        % color of links to bibliography
          filecolor=magenta,      % color of file links
           urlcolor=blue           % color of external links
}
